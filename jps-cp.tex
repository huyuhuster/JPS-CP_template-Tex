\documentclass{jps-cp}
\usepackage{txfonts} %Please comment out this line unless the txfonts package is availabe in your LaTeX system.

\title{XYZ states at BESIII}

\author{Y. \textsc{Hu}$^{1}$}

\inst{$^{1}$Institute of High Energy Physics, Beijing 100049, China}

\email{huyu@ihep.ac.cn}

\recdate{January 18, 2019}



%\abst{ Using the $e^{+}e^{-}$ annihilation data collected above 3.8 GeV with the BESIII detector, we observed the decay $Y(4260) \rightarrow \gamma X(3872)$, and new decay modes $X(3872) \rightarrow \gamma \chi_{c1}$. Cross sections for $e^{+}e^{-} \rightarrow \pi^{+}\pi^{-} J/\psi$, $\pi^{+}\pi^{-}h_{c}$, $\pi^{+}\pi^{-} \psi(2S)$ and $\pi^{+}D^{0}D^{*-}$ are measured, and two structures are observed near 4.2 and 4.4 GeV. Charged  states of $Z_{c}(3900)^{+}$ and $Z_{c}(4020)^{+}$ were observed in $e^{+}e^{-} \rightarrow \pi^{+}\pi^{-} J/\psi$, $\pi^{+}D^{0}D^{*-}$ and $e^{+}e^{-} \rightarrow \pi^{+}\pi^{-} h_{c}$, $\pi^{+}D^{*0}D^{*-}$ processes, respectively. Their neutral partners were also observed, thus the isospin vector is established. The spin and parity of $Z_{c}(3900)^{\pm}$ is determined as $1^{+}$ with statistical significance larger than 7$\sigma$. New decay mode of $Z_{c}(3900)$ are also investigated.}


\abst{ The BESIII experiment has collected about 12 fb$^{-1}$ luminosity data samples above 3.8 GeV. In this talk, we present the recent result on XYZ states at BESIII, including production and decay of the $X(3872)$ states, the precision cross-section measurement for Y states, the establishment of two isospin triplets of $Z_{c}$ states, the  $J^{P}$ determination of $Z_{c}(3900)$ as well as the evidence of new $Z_{c}(3900)^{\pm}$ decay.
}
\kword{Charmonium, Charmoniumlike States, Exotic states, \ldots}

\begin{document}
\maketitle

\section{Introduction}



Except the mesons(composed of one quark and one antiquark) and baryons (composed of three quarks) in the conventional quark model, more types
of hadrons, such as tetra-quarks, penta-quarks and glueball, are also allowed by QCD~\cite{exotics_theory}. The $c\bar{c}$ states can successfully described using potential models. Especially below the open-charm threshold, excellent agreement is achieved between theory and experiment, all the predicted states have been observed with the expected properties. But above the open-charm threshold, there are still many predicted states that have not yet been discovered, and, a lot of unexpected states (called charmoniumlike states or XYZ particles) have been observed in final states with a charmonium and some light hadrons since 2003~\cite{XYZ_review}. They could be candidates for charmonium states, however, there are also strange properties shown from them which make them more like exotic states rather than conventional mesons~\cite{XYZ_review}.

The BESIII experiment at the Beijing Electron Positron Collider (BEPCII) provided a significant contribution to the charmonium-like state spectroscopy thanks to the large data samples collected in the center of mass (CM) energy region between 3.8 and 4.6 GeV. In this proceeding,  we present the most recent results on the study of "XYZ" states at BESIII.



\section{The X states at BESIII}
The first observed $XYZ$ states $X(3872)$ has been discovered by Belle in the spectrum $\pi^{+}\pi^{-}J/\psi$ in the $B^{\pm} \rightarrow K^{\pm} \pi^{+}\pi^{-}J/\psi$ decays~\cite{X3872Belle}, and confirmed by many other experiment~\cite{X3872CDF,X3872D0,X3872BABAR}.
At BESIII, we study the process $e^{+}e^{-} \rightarrow \gamma X(3872) \rightarrow \gamma \pi^{+} \pi^{-} J/\psi$ and observe for the first time  at CM energies from 4.009 to 4.420 GeV~\cite{gammaX3872BESIII}. The mass of $X(3872)$ was measured to be $3871.9 \pm 0.7 \pm 0.2$ MeV, which is consistent with that measured by Belle. Cross sections are measured as shown in~Fig.\ref{X3872}. The lineshape can be described by the $Y(4260)$, which strongly support the existence of the radiative transition process $Y(4260) \rightarrow \gamma X(3872)$.

\begin{figure}[tbh]
\centering
\includegraphics[width=0.286\textwidth]{fig/X3872BESIII_mass.eps}
\includegraphics[width=0.3\textwidth]{fig/X3872BESIII_cross.eps}
\includegraphics[width=0.302\textwidth]{fig/X3872toChic1.eps}
\caption{The $\pi^{+}\pi^{-}J/\psi$ invariant mass distribution (left). Distribution of measured cross section of $e^{+}e^{-} \rightarrow \gamma X(3872) \rightarrow \gamma \pi^{+}\pi^{-}J/\psi$.}
\label{X3872}
\end{figure}

 Other decay channels of $X(3872)$ are also investigated. Using about 9.0 pb$^{-1}$ data with $E_{CM}$  between 4.15 and 4.30 GeV, BESIII preliminarily observed $X(3872)\rightarrow \pi^{0} \chi_{c1}$ decay with a significance of 5.2$\sigma$ as shown in~Fig.\ref{X3872}. The large value for the ratio $\mathcal{B}(X(3872)\rightarrow \pi^{0} \chi_{c1}) / \mathcal{B}(X(3872)\rightarrow \pi^{0} J/\psi) $  disfavors the $\chi_{c1}(2P)$ interpretation of the $X(3872)$~\cite{X3872aschicJ2P}.

\section{The Y states at BESIII}
$Y(4260)$ is the first $Y$ state observed by BABAR experiment in the ISR process $e^{+}e^{-} \rightarrow \gamma \pi^{+}\pi^{-} J/\psi$ \cite{Y4260BABAR}, and then confirmed by Belle~\cite{Y4260Belle} and CLEO~\cite{Y4260CLEO}. Belle also claimed the discovery of another resonance labeled $Y(4008)$. Soon after, BABAR searched for the decays $Y(4260)\rightarrow\pi^{+}\pi^{-}\psi^{'}$, and they observed $Y(4360)$ and $Y(4660)$~\cite{Y4360BABAR}, soon confirmed by Belle~\cite{Y4360Belle}.


\begin{figure}[tbh]
\centering
\includegraphics[width=0.275\textwidth]{fig/pipiJpsi_BESIII_XYZ.eps}
\includegraphics[width=0.253\textwidth]{fig/pipihcBESIII_cross.eps}
\includegraphics[width=0.22\textwidth]{fig/pipipsi3686BESIII_cross.eps}
\includegraphics[width=0.22\textwidth]{fig/Y4630_refit.eps}
\caption{The $\pi^{+}\pi^{-}J/\psi$ invariant mass distribution (left). Distribution of measured cross section of $e^{+}e^{-} \rightarrow \gamma X(3872) \rightarrow \gamma \pi^{+}\pi^{-}J/\psi$.}
\label{Ystates}
\end{figure}

Base on the "high luminosity" data set ($>$ 40 pb$^{-1}$ at each CM energy, dubbed "XYZ data", which dominates the precision of this measurement), and a "low luminosity" data set (7-9 pb$^{-1}$ at each CM energy, dubbed "scan data"), BESIII perform a precision measurement of the cross-section $\sigma(e^{+}e^{-}\rightarrow\pi^{+}\pi^{-}J/\psi)$ alone the energy~\cite{pipiJpsiBESIII}. The signal yields are determined using an unbinned maximum-likelihood fit for XYZ data and a simple counting method for scan data, respectively.   A binned maximum likelihood fit is performed simultaneously to the measured cross section of the XYZ data with Gaussian uncertainties and the scan data with Poisson uncertainties as shown in Fig.~\ref{Ystates}. The cross-section appears inconsistent with a single peak just for the $Y(4260)$, two resonances to describe two peaks is favored over one by the data at high statistical significance $>$ 7$\sigma$.  The mass and width of the first resonance are measured as (4222.0 $\pm$ 3.1 $\pm$ 1.4) MeV/$c^{2}$ and (44.1 $\pm$ 4.3 $\pm$ 2.0) MeV which is consistent with that of the $Y(4260)$ reported by BABAR, CLEO, and Belle, but much narrower. The mass and width of the second resonance are measured as (4320.0 $\pm$ 10.4 $\pm$ 7.0) MeV/$c^{2}$,  (101.4$^{+25.3}_{-19.7}$ $\pm$ 10.2) MeV  which is comparable to the $Y(4360)$ resonance reported by Belle and BABAR in $e^{+}e^{-} \rightarrow \pi^{+}\pi^{-}\psi(2S)$~\cite{Y4360Belle,Y4360BABAR}. The $Y(4360)$ is first observed decaying to $J/\psi \pi^{+}\pi^{-}$.

The line shape of $\pi^{+}\pi^{-}\psi(2S)$ measured by Belle gives clear indication of the decaying of $Y(4360)$ $/$ $Y(4600) \rightarrow \pi^{+} \pi^{-} \psi(2S)$. But there is no evidence for $Y(4260)$ present in the data.  BESIII perform a measurement of the cross-section of $e^{+}e^{-}\rightarrow\pi^{+}\pi^{-}\psi(2S)$~\cite{pipipsi3686BESIII} which confirms the line shape for $Y(4360)$ as shown in Fig.~\ref{Ystates}. But another resonance $Y(4220)$ is also needed to describe the data. BESIII also perform the measurement of its neutral isospin $e^{+}e^{-}\rightarrow\pi^{0}\pi^{0}\psi(3686)$~\cite{pi0pi0psi3686BESIII}. The charge and neutral channels are consistent with each other in isospin symmetry.

BESIII also perform a precise  measurement on the cross section of $e^{+}e^{-}\rightarrow\pi^{+}\pi^{-}h_{c}$ at CM ennergies from 3.90 to 4.6 GeV~\cite{pipihcBESIII}. The energy dependent production cross section as shown in Fig.~\ref{Ystates} shows evidence for two resonant structures dubbed "$Y(4220)$" and "$Y(4390)$", respectively. A fit with a coherent sum of two Breit-Wigner functions results in a mass of (4218.4$^{+5.5}_{-4.5}$ $\pm$ 0.9) MeV/$c^{2}$ and a width of (66.0$^{+12.3}_{-8.3}$ $\pm$ 0.4) MeV/$c^{2}$ for the "$Y(4220)$", and a mass of (4391.6$^{+6.3}_{-6.8}$ $\pm$ 1.0) MeV/$c^{2}$ and a width of (139.5$^{+16.2}_{-20.6}$ $\pm$ 0.6) MeV/$c^{2}$ for the "$Y(4390)$".

BESIII also investigate the $Y$ states in the open charm and light hadrons channels. We have performed a precise cross section measurement of the $e^{+}e^{-}\rightarrow\pi^{+}D^{0}D^{*-}$ process~\cite{DDstarpiBESIII_cross}. Two resonant structures are significant observed in this final states. This is the first experimental evidence for open-charm production associated with the Y states. The first resonance is consistent with $Y(4220)$ measured in $e^{+}e^{-}\rightarrow\omega\chi_{c0}$, $e^{+}e^{-}\rightarrow\pi^{+}\pi^{-}J/\psi$, $e^{+}e^{-}\rightarrow\pi^{+}\pi^{-}\psi(2S)$ and  $e^{+}e^{-}\rightarrow\pi^{+}\pi^{-}h_{c}$. But according to recent result, $Y(4220)$ absent in channel $K^{+}_{S}K^{+}\pi^{-}$~\cite{KKpiBESIII_cross} and $K^{0}_{S}K\pi\pi^{0}/\eta$~\cite{KKpipiBESIII_cross}. There  is no evidence of $Y(4220)$ decays to light hadrons yet.

Belle collaboration observed the baryonic decay of $Y(4660)$~\cite{labdacBelle_cross}. Further more, $Y(4660)$ bayonic coupling is 10 times larger than mesonic coupling. $Y(4660)$ likely is a charmed baryonium. Recently, BESIII report the result of the measurement of $e^{+}e^{-} \rightarrow \Lambda_{c}\bar{\Lambda}_{c}$ near threshold~\cite{labdacBESIII_cross} as shown in Fig.~\ref{Ystates}. But BESIII $\sigma(e^{+}e^{-} \rightarrow \Lambda_{c}\bar{\Lambda}_{c})$ flat with a step at threshold, do not agree so much with Belle data. BESIII is going to increase BEPCII maximum energy,  will confirm (or not) if $Y(4660)$ is a charmed baryonium in the future.



\section{The Z states at BESIII}

The $Z_{c}(3900)^{\pm}$ has been discovered by BESIII~\cite{Zc3900piJpsiBESIII}, seen by Belle~\cite{Zc3900piJpsiBelle} too and shortly after confirmed by CLEO-c~\cite{Zc3900piJpsiCLEOc}. $Z_{c}(3900)^{\pm}$ is a manifestly exotic state, at least made of two charm quarks($c\bar{c}$) and two charged light quarks($u\bar{d}$ or $d\bar{u}$), a good candidate of 4-quark states. A neutral partner $Z_{c}(3900)^{0}$, was also observed in the process $e^{+}e^{-}\rightarrow\pi^{0}\pi^{0}J/\psi$ at BESIII~\cite{Zc3900pi0JpsiBESIII}, confirming earlier evidence report by CLEO-c~\cite{Zc3900piJpsiCLEOc}, and establishing an $Z_{c}(300)^{\pm,0}$ isospin triplet. Furthermore, in recent years, BESIII has reported more other $Z_{c}$ states. Such as $Z_{c}(3885)^{\pm,0}$ in $(D\bar{D}^{*})^{\pm,0}$ system~\cite{Zc3885_charge_BESIII,Zc3885_charge_doubleD_BESIII,Zc3885_neutral_BESIII}, $Z_{c}(4020)^{\pm,0}$ in $\pi^{\pm,0}h_{c}$ system~\cite{Zc40200pihcBESIII,Zc40200pi0hcBESIII}, and $Z_{c}(4025)^{\pm,0}$ in $(D^{*}\bar{D}^{*})^{\pm,0}$ system~\cite{Zc4025_charge_BESIII,Zc4025_neutral_BESIII}. 
BESIII also observed a prominent narrow structure in the system $\pi^{\pm} \psi^{'}$~\cite{pipipsi3686BESIII} as well as a similar structure in the system $\pi^{0} \psi^{'}$~\cite{pi0pi0psi3686BESIII}.  But the fit couldn't describe the data well, a future larger statistics sample of data and theoretical input are needed.

\begin{figure}[tbh]
\centering
%\includegraphics[width=0.9\textwidth]{fig/ZcStates.eps}
\includegraphics[width=0.31\textwidth]{fig/JPZc3900BESIII_fit_piJpsi.eps}
\includegraphics[width=0.208\textwidth]{fig/JPZc3900BESIII_JP_Zc.eps}
\includegraphics[width=0.332\textwidth]{fig/Zc3900RhoEtacBESIII.eps}
\caption{The $\pi^{+}\pi^{-}J/\psi$ invariant mass distribution (left). Distribution of measured cross section of $e^{+}e^{-} \rightarrow \gamma X(3872) \rightarrow \gamma \pi^{+}\pi^{-}J/\psi$.}
\label{ZcStates}
\end{figure}



Recently, to determine the spin and parity of the $Z_{c}$, BESIII performed a partial wave analysis (PWA) of the process $e^{+}e^{-}\rightarrow\pi^{+}\pi^{-} J\psi$~\cite{JPZc3900_BESIII} as shown in Fig.~\ref{ZcStates}. The $J^{P}$ of $Z_{c}(3900)^{\pm}$ was determined to be $1^{+}$ with a statistical significance larger than 7$\sigma$ over other quantum numbers.

The relative decay rate of $Z_{c}^{(')} \rightarrow \rho \eta_{c}$ to $\pi J/\psi(\pi h_{c}) $ may provide an important hint to experimentally distinguish the tetra-quarks (Type-1) and molecular interpretation of $Z_{c}$~\cite{Zc3900rhoetac}. Very recently, BESIII report the preliminary result on the first evidence of $Z_{c}(3900)^{\pm} \rightarrow \rho^{\pm} \eta_{c}$ with statistical significance of 4.3$\sigma$ at $\sqrt{s}$ = 4.23 GeV. No significant signal of $Z_{c}(4020)^{\pm} \rightarrow \rho^{\pm} \eta_{c}$ is found. The measured ratio $R_{Z_{c}(3900)} = \mathcal{B}(Z_{c}(3900)^{\pm} \rightarrow \rho \eta_{c}) / \mathcal{B}(Z_{c}(3900)^{\pm} \rightarrow \pi^{\pm} J/\psi )$ is closer to the calculation of the type-II tetraquark model than those of the other two models.

\section{Summary}
The recent results of XYZ states at BESIII collaboration based on large luminosity data samples collected above charm threshold are presented.  A large number of Z states has been discovered in charmonium and open-charm decays. Several
 isospin triple has been established.  Quantum number of $Z_{c}(3900)^{\pm}$ was determined. The evidence of new decay mode $Z_{c}(3900)^{\pm} \rightarrow \rho^{\pm} \eta_{c}$ was observed. possible decay of the $Y$ into $X(3872)$ (radiative) and new decays of $X(3872)$ has been observed. $Y(4220)$ and $Y(4390)$ have been observed in several decays while the $Y(4660)$ shows a more puzzling behavior.  The nature of these states is still unclear. BESIII will collect more data and plan to increase the beam energy, which will help to resolve the $XYZ$ puzzle.


%\appendix
%\section{}


\begin{thebibliography}{9}

\bibitem{exotics_theory}  E. Klempt and A. Zaitsev, Phys. Rept. \textbf{454}, 1 (2007).

\bibitem{XYZ_review} N. Brambilla {\em et al.}, Eur. Phys. J. C \textbf{71}, 1534 (2011).

\bibitem{X3872Belle} S.-K. Choi {\em et al.} (Belle Collaboration), Phys. Rev. Lett. \textbf{91}, 262001 (2003).
\bibitem{X3872CDF} D. Acosta {\em et al.} (CDF II Collaboration), Phys. Rev. Lett. \textbf{93}, 072001 (2004).
\bibitem{X3872D0} V. M. Abazov {\em et al.} (D0 Collaboration), Phys. Rev. Lett. \textbf{93}, 162002 (2004).
\bibitem{X3872BABAR} B. Aubert {\em et al.} (BABAR Collaboration), Phys. Rev. D \textbf{71}, 071103 (2005).
\bibitem{gammaX3872BESIII} M. Ablikim {\em et al.} (BESIII Collaboration), Phys. Rev. Lett. \textbf{112}, 092001 (2014).
\bibitem{X3872aschicJ2P} S. Dubynskiy and M. B. Voloshin Phys. Rev. D \textbf{77}, 014013 (2008).


\bibitem{Y4260BABAR} B. Aubert {\em et al.} (BABAR Collaboration), Phys. Rev. Lett. \textbf{95}, 142001 (2005).
\bibitem{Y4260Belle} C. Z. Yuan {\em et al.} (Belle Collaboration), Phys. Rev. Lett. \textbf{99}, 182004 (2007).
\bibitem{Y4260CLEO} Q. He {\em et al.} (CLEO Collaboration), Phys. Rev. D \textbf{74}, 091104(R) (2006).
\bibitem{pipiJpsiBESIII} M. Ablikim {\em et al.} (BESIII Collaboration), Phys. Rev. Lett. \textbf{118}, 092001 (2017).
\bibitem{Y4360BABAR} B. Aubert {\em et al.} (BABAR Collaboration), Phys. Rev. Lett. \textbf{98}, 212001 (2007).
\bibitem{Y4360Belle} X. L. Wang {\em et al.} (Belle Collaboration), Phys. Rev. Lett. \textbf{99}, 142002 (2007).
\bibitem{pipipsi3686BESIII} M. Ablikim {\em et al.} (BESIII Collaboration), Phys. Rev. Lett. \textbf{96}, 032004 (2017).
\bibitem{pi0pi0psi3686BESIII} M. Ablikim {\em et al.} (BESIII Collaboration), Phys. Rev. D \textbf{97}, 052001 (2018).
\bibitem{pipihcBESIII} M. Ablikim {\em et al.} (BESIII Collaboration), Phys. Rev. Lett. \textbf{118}, 092002 (2017).
\bibitem{DDstarpiBESIII_cross}  M. Ablikim {\em et al.} (BESIII Collaboration), arXiv:1808.02847.
\bibitem{KKpiBESIII_cross} M. Ablikim {\em et al.} (BESIII Collaboration), arXiv:1808.08733.
\bibitem{KKpipiBESIII_cross} M. Ablikim {\em et al.} (BESIII Collaboration), arXiv:1810.09395.
\bibitem{labdacBESIII_cross} M. Ablikim {\em et al.} (BESIII Collaboration), Phys. Rev. Lett. \textbf{120}, 132001 (2018).
\bibitem{labdacBelle_cross} G. Pakhlova {\em et al.} (The Belle Collaboration), Phys. Rev. Lett. \textbf{101}, 172001 (2008).


\bibitem{Zc3900piJpsiBESIII} M. Ablikim {\em et al.} (BESIII Collaboration), Phys. Rev. Lett. \textbf{110}, 252001 (2013).
\bibitem{Zc3900piJpsiBelle} Z. Q. Liu {\em et al.} (Belle Collaboration), Phys. Rev. Lett. \textbf{110}, 252002 (2013).
\bibitem{Zc3900piJpsiCLEOc} T. Xiao, S. Dobbs, A. Tomaradze and K. K. Seth, Phys. Lett. B \textbf{727}, 366 (2013).
\bibitem{Zc3900pi0JpsiBESIII} M. Ablikim {\em et al.} (BESIII Collaboration), Phys. Rev. Lett. \textbf{115}, 112003 (2015).
\bibitem{Zc3885_charge_BESIII} M. Ablikim {\em et al.} (BESIII Collaboration), Phys. Rev. Lett. \textbf{112}, 022001 (2014).
\bibitem{Zc3885_charge_doubleD_BESIII} M. Ablikim {\em et al.} (BESIII Collaboration), Phys. Rev. D \textbf{92}, 092006 (2015).
\bibitem{Zc3885_neutral_BESIII} M. Ablikim {\em et al.} (BESIII Collaboration), Phys. Rev. Lett. \textbf{115}, 222002 (2015).
\bibitem{Zc40200pihcBESIII} M. Ablikim {\em et al.} (BESIII Collaboration), Phys. Rev. Lett. \textbf{111}, 242001 (2013).
\bibitem{Zc40200pi0hcBESIII} M. Ablikim {\em et al.} (BESIII Collaboration), Phys. Rev. Lett. \textbf{113}, 212002 (2014).
\bibitem{Zc4025_charge_BESIII} M. Ablikim {\em et al.} (BESIII Collaboration), Phys. Rev. Lett. \textbf{112}, 132001 (2014).
\bibitem{Zc4025_neutral_BESIII} M. Ablikim {\em et al.} (BESIII Collaboration), Phys. Rev. Lett. \textbf{115}, 182002 (2015).
\bibitem{JPZc3900_BESIII} M. Ablikim {\em et al.} (BESIII Collaboration), Phys. Rev. Lett. \textbf{119}, 072001 (2017).
\bibitem{Zc3900rhoetac} A. Esposito, A. L. Guerrieri and A. Pilloni, Phys. Lett. B \textbf{746}, 194 (2015).
\end{thebibliography}


\end{document}


%\begin{table}[tbh]
%\caption{Captions to tables and figures should be sentences.}
%\label{t1}
%\begin{tabular}{ll}
%\hline
%AAA & BBB \\
%CCC & DDD \\
%\hline
%\end{tabular}
%\end{table}


%Use the \verb|\appendix| command if you need an appendix(es). The \verb|\section| command should follow even though there is no title for the appendix (see above in the source of this file).
%
%
%Label figures, tables, and equations appropriately using the \verb|\label| command, and use the \verb|\ref| command to cite them in the text as ``\verb|as shown in Fig. \ref{f1}|". This automatically labels the numbers in numerical order.
%
%The \verb|minipage| environment can be used to place figures horizontally.
%
%\begin{equation}
%E = mc^{2}
%\label{e1}
%\end{equation}
